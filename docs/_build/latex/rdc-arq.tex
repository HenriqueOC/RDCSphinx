%% Generated by Sphinx.
\def\sphinxdocclass{report}
\documentclass[letterpaper,10pt,english]{sphinxmanual}
\ifdefined\pdfpxdimen
   \let\sphinxpxdimen\pdfpxdimen\else\newdimen\sphinxpxdimen
\fi \sphinxpxdimen=.75bp\relax

\PassOptionsToPackage{warn}{textcomp}
\usepackage[utf8]{inputenc}
\ifdefined\DeclareUnicodeCharacter
% support both utf8 and utf8x syntaxes
  \ifdefined\DeclareUnicodeCharacterAsOptional
    \def\sphinxDUC#1{\DeclareUnicodeCharacter{"#1}}
  \else
    \let\sphinxDUC\DeclareUnicodeCharacter
  \fi
  \sphinxDUC{00A0}{\nobreakspace}
  \sphinxDUC{2500}{\sphinxunichar{2500}}
  \sphinxDUC{2502}{\sphinxunichar{2502}}
  \sphinxDUC{2514}{\sphinxunichar{2514}}
  \sphinxDUC{251C}{\sphinxunichar{251C}}
  \sphinxDUC{2572}{\textbackslash}
\fi
\usepackage{cmap}
\usepackage[T1]{fontenc}
\usepackage{amsmath,amssymb,amstext}
\usepackage{babel}



\usepackage{times}
\expandafter\ifx\csname T@LGR\endcsname\relax
\else
% LGR was declared as font encoding
  \substitutefont{LGR}{\rmdefault}{cmr}
  \substitutefont{LGR}{\sfdefault}{cmss}
  \substitutefont{LGR}{\ttdefault}{cmtt}
\fi
\expandafter\ifx\csname T@X2\endcsname\relax
  \expandafter\ifx\csname T@T2A\endcsname\relax
  \else
  % T2A was declared as font encoding
    \substitutefont{T2A}{\rmdefault}{cmr}
    \substitutefont{T2A}{\sfdefault}{cmss}
    \substitutefont{T2A}{\ttdefault}{cmtt}
  \fi
\else
% X2 was declared as font encoding
  \substitutefont{X2}{\rmdefault}{cmr}
  \substitutefont{X2}{\sfdefault}{cmss}
  \substitutefont{X2}{\ttdefault}{cmtt}
\fi


\usepackage[Bjarne]{fncychap}
\usepackage{sphinx}

\fvset{fontsize=\small}
\usepackage{geometry}


% Include hyperref last.
\usepackage{hyperref}
% Fix anchor placement for figures with captions.
\usepackage{hypcap}% it must be loaded after hyperref.
% Set up styles of URL: it should be placed after hyperref.
\urlstyle{same}

\addto\captionsenglish{\renewcommand{\contentsname}{Conteúdo:}}

\usepackage{sphinxmessages}
\setcounter{tocdepth}{1}



\title{RDC\sphinxhyphen{}ARQ}
\date{Jul 09, 2020}
\release{}
\author{Henrique Oliveira Costa}
\newcommand{\sphinxlogo}{\vbox{}}
\renewcommand{\releasename}{}
\makeindex
\begin{document}

\pagestyle{empty}
\sphinxmaketitle
\pagestyle{plain}
\sphinxtableofcontents
\pagestyle{normal}
\phantomsection\label{\detokenize{index::doc}}


Visão geral

A elaboração de estudos que indiquem as soluções viáveis de implantação de um repositório arquivístico digital confiável \sphinxhyphen{} RDC \sphinxhyphen{} Arq para o TJDFT

Objetivo Geral

Apresentar alternativas de soluções para implantação de um repositório arquivístico digital confiável \sphinxhyphen{} RDC \sphinxhyphen{} Arq, integrado com solução de gerenciamento documental, tornando possível a elaboração e disseminação de modelo tecnológico de mapeamento, preservação e disseminação das informações associadas.


\chapter{Classes}
\label{\detokenize{modules:classes}}\label{\detokenize{modules::doc}}

\section{ClasseGeral module}
\label{\detokenize{ClasseGeral:module-ClasseGeral}}\label{\detokenize{ClasseGeral:classegeral-module}}\label{\detokenize{ClasseGeral::doc}}\index{module@\spxentry{module}!ClasseGeral@\spxentry{ClasseGeral}}\index{ClasseGeral@\spxentry{ClasseGeral}!module@\spxentry{module}}\index{Geral (class in ClasseGeral)@\spxentry{Geral}\spxextra{class in ClasseGeral}}

\begin{fulllineitems}
\phantomsection\label{\detokenize{ClasseGeral:ClasseGeral.Geral}}\pysiglinewithargsret{\sphinxbfcode{\sphinxupquote{class }}\sphinxcode{\sphinxupquote{ClasseGeral.}}\sphinxbfcode{\sphinxupquote{Geral}}}{\emph{\DUrole{n}{sistema}}}{}
Bases: \sphinxcode{\sphinxupquote{object}}

Executa cada uma das etapas necessárias para a criação do pacote OAIS.
\index{conexaoBanco() (ClasseGeral.Geral method)@\spxentry{conexaoBanco()}\spxextra{ClasseGeral.Geral method}}

\begin{fulllineitems}
\phantomsection\label{\detokenize{ClasseGeral:ClasseGeral.Geral.conexaoBanco}}\pysiglinewithargsret{\sphinxbfcode{\sphinxupquote{conexaoBanco}}}{}{}
Executa a função de conexão com o banco de dados do DJe.

\end{fulllineitems}

\index{geradorConsultas() (ClasseGeral.Geral method)@\spxentry{geradorConsultas()}\spxextra{ClasseGeral.Geral method}}

\begin{fulllineitems}
\phantomsection\label{\detokenize{ClasseGeral:ClasseGeral.Geral.geradorConsultas}}\pysiglinewithargsret{\sphinxbfcode{\sphinxupquote{geradorConsultas}}}{}{}
Executa a função de consulta dos metadados no banco de dados.

\end{fulllineitems}

\index{geradorMetadatas() (ClasseGeral.Geral method)@\spxentry{geradorMetadatas()}\spxextra{ClasseGeral.Geral method}}

\begin{fulllineitems}
\phantomsection\label{\detokenize{ClasseGeral:ClasseGeral.Geral.geradorMetadatas}}\pysiglinewithargsret{\sphinxbfcode{\sphinxupquote{geradorMetadatas}}}{}{}
Executa a função de extração de metadados.

\end{fulllineitems}

\index{infoArchivematica() (ClasseGeral.Geral method)@\spxentry{infoArchivematica()}\spxextra{ClasseGeral.Geral method}}

\begin{fulllineitems}
\phantomsection\label{\detokenize{ClasseGeral:ClasseGeral.Geral.infoArchivematica}}\pysiglinewithargsret{\sphinxbfcode{\sphinxupquote{infoArchivematica}}}{}{}
Executa a função de extrair as informações do Archivematica.

\end{fulllineitems}

\index{metadatas (ClasseGeral.Geral attribute)@\spxentry{metadatas}\spxextra{ClasseGeral.Geral attribute}}

\begin{fulllineitems}
\phantomsection\label{\detokenize{ClasseGeral:ClasseGeral.Geral.metadatas}}\pysigline{\sphinxbfcode{\sphinxupquote{metadatas}}\sphinxbfcode{\sphinxupquote{ = 0}}}
\end{fulllineitems}

\index{recuperaXML() (ClasseGeral.Geral method)@\spxentry{recuperaXML()}\spxextra{ClasseGeral.Geral method}}

\begin{fulllineitems}
\phantomsection\label{\detokenize{ClasseGeral:ClasseGeral.Geral.recuperaXML}}\pysiglinewithargsret{\sphinxbfcode{\sphinxupquote{recuperaXML}}}{}{}
Mapeia o obtêm o arquivo XML.

\end{fulllineitems}

\index{runAll() (ClasseGeral.Geral method)@\spxentry{runAll()}\spxextra{ClasseGeral.Geral method}}

\begin{fulllineitems}
\phantomsection\label{\detokenize{ClasseGeral:ClasseGeral.Geral.runAll}}\pysiglinewithargsret{\sphinxbfcode{\sphinxupquote{runAll}}}{}{}
Executa todas as funções necessárias para funcionamento do barramento.

\end{fulllineitems}

\index{steve() (ClasseGeral.Geral method)@\spxentry{steve()}\spxextra{ClasseGeral.Geral method}}

\begin{fulllineitems}
\phantomsection\label{\detokenize{ClasseGeral:ClasseGeral.Geral.steve}}\pysiglinewithargsret{\sphinxbfcode{\sphinxupquote{steve}}}{}{}
Executa a função job.

\end{fulllineitems}


\end{fulllineitems}



\section{ConexaoBancoDje module}
\label{\detokenize{ConexaoBancoDje:module-ConexaoBancoDje}}\label{\detokenize{ConexaoBancoDje:conexaobancodje-module}}\label{\detokenize{ConexaoBancoDje::doc}}\index{module@\spxentry{module}!ConexaoBancoDje@\spxentry{ConexaoBancoDje}}\index{ConexaoBancoDje@\spxentry{ConexaoBancoDje}!module@\spxentry{module}}\index{conexaomBanco() (in module ConexaoBancoDje)@\spxentry{conexaomBanco()}\spxextra{in module ConexaoBancoDje}}

\begin{fulllineitems}
\phantomsection\label{\detokenize{ConexaoBancoDje:ConexaoBancoDje.conexaomBanco}}\pysiglinewithargsret{\sphinxcode{\sphinxupquote{ConexaoBancoDje.}}\sphinxbfcode{\sphinxupquote{conexaomBanco}}}{\emph{\DUrole{n}{root}}}{}
Realizar a conexão com o banco de dados do DJe.
\begin{quote}\begin{description}
\item[{Parameters}] \leavevmode
\sphinxstyleliteralstrong{\sphinxupquote{conexao}} (\sphinxstyleliteralemphasis{\sphinxupquote{xml.Element}}) \textendash{} Dados necessários para a conexão com o banco de dados.

\end{description}\end{quote}

\end{fulllineitems}



\section{ConexaoBancoPje module}
\label{\detokenize{ConexaoBancoPje:module-ConexaoBancoPje}}\label{\detokenize{ConexaoBancoPje:conexaobancopje-module}}\label{\detokenize{ConexaoBancoPje::doc}}\index{module@\spxentry{module}!ConexaoBancoPje@\spxentry{ConexaoBancoPje}}\index{ConexaoBancoPje@\spxentry{ConexaoBancoPje}!module@\spxentry{module}}\index{conexaoComBanco() (in module ConexaoBancoPje)@\spxentry{conexaoComBanco()}\spxextra{in module ConexaoBancoPje}}

\begin{fulllineitems}
\phantomsection\label{\detokenize{ConexaoBancoPje:ConexaoBancoPje.conexaoComBanco}}\pysiglinewithargsret{\sphinxcode{\sphinxupquote{ConexaoBancoPje.}}\sphinxbfcode{\sphinxupquote{conexaoComBanco}}}{\emph{\DUrole{n}{algo}}}{}
\end{fulllineitems}



\section{ConsultaDje module}
\label{\detokenize{ConsultaDje:module-ConsultaDje}}\label{\detokenize{ConsultaDje:consultadje-module}}\label{\detokenize{ConsultaDje::doc}}\index{module@\spxentry{module}!ConsultaDje@\spxentry{ConsultaDje}}\index{ConsultaDje@\spxentry{ConsultaDje}!module@\spxentry{module}}\index{consultas() (in module ConsultaDje)@\spxentry{consultas()}\spxextra{in module ConsultaDje}}

\begin{fulllineitems}
\phantomsection\label{\detokenize{ConsultaDje:ConsultaDje.consultas}}\pysiglinewithargsret{\sphinxcode{\sphinxupquote{ConsultaDje.}}\sphinxbfcode{\sphinxupquote{consultas}}}{\emph{\DUrole{n}{root}}, \emph{\DUrole{n}{cursor}}}{}
Realiza a consulta SQL no banco de dados do sistema e extrai a consulta, salvando em uma tabela.
\begin{quote}\begin{description}
\item[{Parameters}] \leavevmode\begin{itemize}
\item {} 
\sphinxstyleliteralstrong{\sphinxupquote{xmlMetadados}} (\sphinxstyleliteralemphasis{\sphinxupquote{xml.Element}}) \textendash{} XML extraído do arquivo dje.xml.

\item {} 
\sphinxstyleliteralstrong{\sphinxupquote{cursor}} (\sphinxstyleliteralemphasis{\sphinxupquote{python.Cursor}}) \textendash{} Classe utilizada para executar comandos SQL.

\item {} 
\sphinxstyleliteralstrong{\sphinxupquote{enviosSequenciais}} (\sphinxstyleliteralemphasis{\sphinxupquote{int}}) \textendash{} Informa quantas consultas serão feitas no banco de dados.

\end{itemize}

\item[{Returns}] \leavevmode
tabela1

\end{description}\end{quote}

\end{fulllineitems}

\index{consultas2() (in module ConsultaDje)@\spxentry{consultas2()}\spxextra{in module ConsultaDje}}

\begin{fulllineitems}
\phantomsection\label{\detokenize{ConsultaDje:ConsultaDje.consultas2}}\pysiglinewithargsret{\sphinxcode{\sphinxupquote{ConsultaDje.}}\sphinxbfcode{\sphinxupquote{consultas2}}}{\emph{\DUrole{n}{consulta}}, \emph{\DUrole{n}{cursor}}}{}
Extrai os metadados da primeira consulta.
\begin{quote}\begin{description}
\item[{Parameters}] \leavevmode\begin{itemize}
\item {} 
\sphinxstyleliteralstrong{\sphinxupquote{consulta}} (\sphinxstyleliteralemphasis{\sphinxupquote{string}}) \textendash{} Comando a ser utilizado no banco de dados.

\item {} 
\sphinxstyleliteralstrong{\sphinxupquote{cursor}} (\sphinxstyleliteralemphasis{\sphinxupquote{python.Cursor}}) \textendash{} Classe utilizada para executar comandos SQL.

\end{itemize}

\item[{Returns}] \leavevmode
cursor.fetchall(), columns

\end{description}\end{quote}

\end{fulllineitems}

\index{verificadorDeID() (in module ConsultaDje)@\spxentry{verificadorDeID()}\spxextra{in module ConsultaDje}}

\begin{fulllineitems}
\phantomsection\label{\detokenize{ConsultaDje:ConsultaDje.verificadorDeID}}\pysiglinewithargsret{\sphinxcode{\sphinxupquote{ConsultaDje.}}\sphinxbfcode{\sphinxupquote{verificadorDeID}}}{}{}
Verifica pelo ID do banco de dados o último diário para ser dado continuidade.
\begin{quote}\begin{description}
\item[{Returns}] \leavevmode
null

\end{description}\end{quote}

\end{fulllineitems}



\section{ConsultaPje module}
\label{\detokenize{ConsultaPje:module-ConsultaPje}}\label{\detokenize{ConsultaPje:consultapje-module}}\label{\detokenize{ConsultaPje::doc}}\index{module@\spxentry{module}!ConsultaPje@\spxentry{ConsultaPje}}\index{ConsultaPje@\spxentry{ConsultaPje}!module@\spxentry{module}}\index{consultas() (in module ConsultaPje)@\spxentry{consultas()}\spxextra{in module ConsultaPje}}

\begin{fulllineitems}
\phantomsection\label{\detokenize{ConsultaPje:ConsultaPje.consultas}}\pysiglinewithargsret{\sphinxcode{\sphinxupquote{ConsultaPje.}}\sphinxbfcode{\sphinxupquote{consultas}}}{\emph{\DUrole{n}{root}}, \emph{\DUrole{n}{cursor}}}{}
\end{fulllineitems}



\section{EnvioArquivematica module}
\label{\detokenize{EnvioArquivematica:module-EnvioArquivematica}}\label{\detokenize{EnvioArquivematica:envioarquivematica-module}}\label{\detokenize{EnvioArquivematica::doc}}\index{module@\spxentry{module}!EnvioArquivematica@\spxentry{EnvioArquivematica}}\index{EnvioArquivematica@\spxentry{EnvioArquivematica}!module@\spxentry{module}}\index{envioArchivematica() (in module EnvioArquivematica)@\spxentry{envioArchivematica()}\spxextra{in module EnvioArquivematica}}

\begin{fulllineitems}
\phantomsection\label{\detokenize{EnvioArquivematica:EnvioArquivematica.envioArchivematica}}\pysiglinewithargsret{\sphinxcode{\sphinxupquote{EnvioArquivematica.}}\sphinxbfcode{\sphinxupquote{envioArchivematica}}}{\emph{\DUrole{n}{path}}, \emph{\DUrole{n}{infoServer}}, \emph{\DUrole{n}{raiz}}, \emph{\DUrole{n}{nomeArquivo}}, \emph{\DUrole{n}{identificador}}, \emph{\DUrole{n}{pastaRaiz}}}{}
Executar o envio do pacote OAIS para o Archivematica e registro no log de pacotes enviados.
\begin{quote}\begin{description}
\item[{Parameters}] \leavevmode\begin{itemize}
\item {} 
\sphinxstyleliteralstrong{\sphinxupquote{path}} (\sphinxstyleliteralemphasis{\sphinxupquote{str}}) \textendash{} Diretório interno do pacote OAIS.

\item {} 
\sphinxstyleliteralstrong{\sphinxupquote{infoServer}} (\sphinxstyleliteralemphasis{\sphinxupquote{list}}) \textendash{} Informações necessárias para conexão com o Banco de Dados.

\item {} 
\sphinxstyleliteralstrong{\sphinxupquote{raiz}} (\sphinxstyleliteralemphasis{\sphinxupquote{str}}) \textendash{} Diretório do pacote OAIS.

\item {} 
\sphinxstyleliteralstrong{\sphinxupquote{nomeArquivo}} (\sphinxstyleliteralemphasis{\sphinxupquote{str}}) \textendash{} Nome do pacote OAIS.

\item {} 
\sphinxstyleliteralstrong{\sphinxupquote{identificador}} (\sphinxstyleliteralemphasis{\sphinxupquote{str}}) \textendash{} ID do diário no banco de dados.

\item {} 
\sphinxstyleliteralstrong{\sphinxupquote{pastaRaiz}} (\sphinxstyleliteralemphasis{\sphinxupquote{str}}) \textendash{} Diretório do barramento.

\end{itemize}

\item[{Returns}] \leavevmode
null

\end{description}\end{quote}

\end{fulllineitems}



\section{Imports module}
\label{\detokenize{Imports:module-Imports}}\label{\detokenize{Imports:imports-module}}\label{\detokenize{Imports::doc}}\index{module@\spxentry{module}!Imports@\spxentry{Imports}}\index{Imports@\spxentry{Imports}!module@\spxentry{module}}

\section{JobDje module}
\label{\detokenize{JobDje:module-JobDje}}\label{\detokenize{JobDje:jobdje-module}}\label{\detokenize{JobDje::doc}}\index{module@\spxentry{module}!JobDje@\spxentry{JobDje}}\index{JobDje@\spxentry{JobDje}!module@\spxentry{module}}\index{job() (in module JobDje)@\spxentry{job()}\spxextra{in module JobDje}}

\begin{fulllineitems}
\phantomsection\label{\detokenize{JobDje:JobDje.job}}\pysiglinewithargsret{\sphinxcode{\sphinxupquote{JobDje.}}\sphinxbfcode{\sphinxupquote{job}}}{\emph{\DUrole{n}{metadatas}}, \emph{\DUrole{n}{infoServer}}}{}
Criação de pasta e organização dos arquivos para a criação do pacote OAIS.
\begin{quote}\begin{description}
\item[{Parameters}] \leavevmode\begin{itemize}
\item {} 
\sphinxstyleliteralstrong{\sphinxupquote{metadatas}} (\sphinxstyleliteralemphasis{\sphinxupquote{tuple}}) \textendash{} Metadados do diário obtidos.

\item {} 
\sphinxstyleliteralstrong{\sphinxupquote{infoServer}} (\sphinxstyleliteralemphasis{\sphinxupquote{list}}) \textendash{} Informações necessárias para conexão com o Banco de Dados.

\end{itemize}

\item[{Returns}] \leavevmode
empty

\end{description}\end{quote}

\end{fulllineitems}



\section{JobPje module}
\label{\detokenize{JobPje:module-JobPje}}\label{\detokenize{JobPje:jobpje-module}}\label{\detokenize{JobPje::doc}}\index{module@\spxentry{module}!JobPje@\spxentry{JobPje}}\index{JobPje@\spxentry{JobPje}!module@\spxentry{module}}\index{job() (in module JobPje)@\spxentry{job()}\spxextra{in module JobPje}}

\begin{fulllineitems}
\phantomsection\label{\detokenize{JobPje:JobPje.job}}\pysiglinewithargsret{\sphinxcode{\sphinxupquote{JobPje.}}\sphinxbfcode{\sphinxupquote{job}}}{\emph{\DUrole{n}{metadatas}}, \emph{\DUrole{n}{infoServer}}}{}
\end{fulllineitems}



\section{MetadatasDje module}
\label{\detokenize{MetadatasDje:module-MetadatasDje}}\label{\detokenize{MetadatasDje:metadatasdje-module}}\label{\detokenize{MetadatasDje::doc}}\index{module@\spxentry{module}!MetadatasDje@\spxentry{MetadatasDje}}\index{MetadatasDje@\spxentry{MetadatasDje}!module@\spxentry{module}}\index{geraMetadatas() (in module MetadatasDje)@\spxentry{geraMetadatas()}\spxextra{in module MetadatasDje}}

\begin{fulllineitems}
\phantomsection\label{\detokenize{MetadatasDje:MetadatasDje.geraMetadatas}}\pysiglinewithargsret{\sphinxcode{\sphinxupquote{MetadatasDje.}}\sphinxbfcode{\sphinxupquote{geraMetadatas}}}{\emph{\DUrole{n}{root}}, \emph{\DUrole{n}{tabela}}, \emph{\DUrole{n}{cursor}}}{}
Extrai os arquivos de metadados.
\begin{quote}\begin{description}
\item[{Parameters}] \leavevmode\begin{itemize}
\item {} 
\sphinxstyleliteralstrong{\sphinxupquote{root}} (\sphinxstyleliteralemphasis{\sphinxupquote{xml.Element}}) \textendash{} XML extraído do arquivo dje.xml.

\item {} 
\sphinxstyleliteralstrong{\sphinxupquote{tabela}} (\sphinxstyleliteralemphasis{\sphinxupquote{dict}}) \textendash{} Metadados extraídos do banco de dados do DJe.

\item {} 
\sphinxstyleliteralstrong{\sphinxupquote{cursor}} (\sphinxstyleliteralemphasis{\sphinxupquote{python.Cursor}}) \textendash{} Classe utilizada para executar comandos SQL.

\end{itemize}

\item[{Returns}] \leavevmode
variaveis, nomeArquivo, PDF

\end{description}\end{quote}

\end{fulllineitems}

\index{valor() (in module MetadatasDje)@\spxentry{valor()}\spxextra{in module MetadatasDje}}

\begin{fulllineitems}
\phantomsection\label{\detokenize{MetadatasDje:MetadatasDje.valor}}\pysiglinewithargsret{\sphinxcode{\sphinxupquote{MetadatasDje.}}\sphinxbfcode{\sphinxupquote{valor}}}{\emph{\DUrole{n}{objeto}}, \emph{\DUrole{n}{coluna}}}{}
Obtêm a chave a ser consultada para obtenção do metadados.
\begin{quote}\begin{description}
\item[{Parameters}] \leavevmode\begin{itemize}
\item {} 
\sphinxstyleliteralstrong{\sphinxupquote{objeto}} (\sphinxstyleliteralemphasis{\sphinxupquote{pyodbc.Row}}) \textendash{} Consulta do banco de dados possuindo as chaves a serem consultadas.

\item {} 
\sphinxstyleliteralstrong{\sphinxupquote{coluna}} (\sphinxstyleliteralemphasis{\sphinxupquote{string}}) \textendash{} Variável que recebe o valor da coluna extraída do banco de dados.

\end{itemize}

\item[{Returns}] \leavevmode
tempValor

\end{description}\end{quote}

\end{fulllineitems}



\section{MetadatasPje module}
\label{\detokenize{MetadatasPje:module-MetadatasPje}}\label{\detokenize{MetadatasPje:metadataspje-module}}\label{\detokenize{MetadatasPje::doc}}\index{module@\spxentry{module}!MetadatasPje@\spxentry{MetadatasPje}}\index{MetadatasPje@\spxentry{MetadatasPje}!module@\spxentry{module}}\index{geraMetadatas() (in module MetadatasPje)@\spxentry{geraMetadatas()}\spxextra{in module MetadatasPje}}

\begin{fulllineitems}
\phantomsection\label{\detokenize{MetadatasPje:MetadatasPje.geraMetadatas}}\pysiglinewithargsret{\sphinxcode{\sphinxupquote{MetadatasPje.}}\sphinxbfcode{\sphinxupquote{geraMetadatas}}}{\emph{\DUrole{n}{root}}, \emph{\DUrole{n}{retornoConsultas}}, \emph{\DUrole{n}{cursor}}}{}
\end{fulllineitems}



\chapter{Indices e tabelas}
\label{\detokenize{index:indices-e-tabelas}}\begin{itemize}
\item {} 
\DUrole{xref,std,std-ref}{genindex}

\item {} 
\DUrole{xref,std,std-ref}{modindex}

\item {} 
\DUrole{xref,std,std-ref}{search}

\end{itemize}


\renewcommand{\indexname}{Python Module Index}
\begin{sphinxtheindex}
\let\bigletter\sphinxstyleindexlettergroup
\bigletter{c}
\item\relax\sphinxstyleindexentry{ClasseGeral}\sphinxstyleindexpageref{ClasseGeral:\detokenize{module-ClasseGeral}}
\item\relax\sphinxstyleindexentry{ConexaoBancoDje}\sphinxstyleindexpageref{ConexaoBancoDje:\detokenize{module-ConexaoBancoDje}}
\item\relax\sphinxstyleindexentry{ConexaoBancoPje}\sphinxstyleindexpageref{ConexaoBancoPje:\detokenize{module-ConexaoBancoPje}}
\item\relax\sphinxstyleindexentry{ConsultaDje}\sphinxstyleindexpageref{ConsultaDje:\detokenize{module-ConsultaDje}}
\item\relax\sphinxstyleindexentry{ConsultaPje}\sphinxstyleindexpageref{ConsultaPje:\detokenize{module-ConsultaPje}}
\indexspace
\bigletter{e}
\item\relax\sphinxstyleindexentry{EnvioArquivematica}\sphinxstyleindexpageref{EnvioArquivematica:\detokenize{module-EnvioArquivematica}}
\indexspace
\bigletter{i}
\item\relax\sphinxstyleindexentry{Imports}\sphinxstyleindexpageref{Imports:\detokenize{module-Imports}}
\indexspace
\bigletter{j}
\item\relax\sphinxstyleindexentry{JobDje}\sphinxstyleindexpageref{JobDje:\detokenize{module-JobDje}}
\item\relax\sphinxstyleindexentry{JobPje}\sphinxstyleindexpageref{JobPje:\detokenize{module-JobPje}}
\indexspace
\bigletter{m}
\item\relax\sphinxstyleindexentry{MetadatasDje}\sphinxstyleindexpageref{MetadatasDje:\detokenize{module-MetadatasDje}}
\item\relax\sphinxstyleindexentry{MetadatasPje}\sphinxstyleindexpageref{MetadatasPje:\detokenize{module-MetadatasPje}}
\end{sphinxtheindex}

\renewcommand{\indexname}{Index}
\printindex
\end{document}